\documentclass{article} % Define la clase del documento, en este caso, un artículo
\usepackage[letterpaper,margin=3cm]{geometry} % Configura el tamaño del papel y los márgenes del documento
\usepackage{graphicx} % Permite la inserción de imágenes
\usepackage[spanish]{babel}% Activar esta configuración para informes en español, ajusta el idioma del documento
\usepackage[usenames]{color} % Permite el uso de colores definidos por nombre en el documento
\usepackage{hyperref} % Habilita enlaces y referencias dentro del documento
\hypersetup{colorlinks=true, linkcolor = black, citecolor= black} % Configura el color de los enlaces y citas
\usepackage{booktabs} % Proporciona comandos para crear tablas de alta calidad
\usepackage{natbib} % Permite el uso de citas y referencias bibliográficas con diferentes estilos
\usepackage{tikz} % Permite la creación de gráficos y diagramas vectoriales directamente en LaTeX
\usepackage{float} % Para controlar la posición de los elementos flotantes, como imágenes, con la opción [H]
\usepackage{diagbox} % Permite crear celdas con líneas diagonales en tablas
\usepackage{listings} % Permite la inclusión y formateo de código fuente en el documento
\usepackage{xcolor} % Paquete para definir y usar colores en el documento
\usepackage{parskip} % Añade espacio entre párrafos en lugar de sangrías
\usepackage{fancyhdr} % Permite personalizar encabezados y pies de página
\usepackage{amsmath} % Proporciona una amplia variedad de entornos y comandos matemáticos

\pagestyle{fancy} % Usa el estilo fancyhdr
\fancyhf{} % Borra todos los encabezados y pies de página
\renewcommand{\headrulewidth}{0pt}
\renewcommand{\footrulewidth}{0pt} % Desactiva la línea horizontal predeterminada en el pie
\setlength{\headheight}{2cm} % Ajusta la altura del encabezado para hacer espacio para la línea
\fancyhead[L]{\raisebox{0.20cm}{\textbf{Proyecto de infraestructura hidráulica}}} % Añade el texto en la parte izquierda del encabezado, subiéndolo ligeramente
\fancyhead[R]{\raisebox{0.1cm}{\includegraphics[width=0.25\linewidth]{LOGO_UNIVERSIDAD.jpg}}} % Añade la imagen en la parte derecha del encabezado y súbela un poco
\fancyhead[C]{\rule{\textwidth}{0.6pt}} % Añade una línea horizontal superior centrada
\fancyfoot[C]{\rule{\textwidth}{0.6pt}} % Añade una línea horizontal en el pie de página centrada
\fancyfoot[R]{\raisebox{-1.5\baselineskip}{\thepage}} % Coloca el número de página a la derecha, con suficiente espacio debajo de la línea
\geometry{top=3cm, bottom=2.5cm} % Ajusta los márgenes superior e inferior

% Definición de colores al estilo Visual Studio Code
\definecolor{codegreen}{rgb}{0.25,0.49,0.48} % Comentarios
\definecolor{codegray}{rgb}{0.5,0.5,0.5} % Números y anotaciones
\definecolor{codepurple}{rgb}{0.58,0,0.82} % Palabras clave
\definecolor{backcolour}{rgb}{0.95,0.95,0.92} % Color de fondo

% Configuración del estilo de las celdas de código
\lstset{
    backgroundcolor=\color{backcolour},   % color de fondo; necesita que el paquete color o xcolor esté cargado
    commentstyle=\color{codegreen},       % estilo de comentarios
    keywordstyle=\color{codepurple},      % estilo de palabras clave
    numberstyle=\tiny\color{codegray},    % estilo de los números de línea
    stringstyle=\color{red},              % estilo de las cadenas de texto
    basicstyle=\ttfamily\small,           % estilo del texto básico
    breakatwhitespace=false,              % ajustes de líneas sólo en espacios en blanco
    breaklines=true,                      % ajustar las líneas si son muy largas
    captionpos=b,                         % posición de la leyenda (abajo)
    keepspaces=true,                      % preserva los espacios en el texto; útil si se usa monoespaciado
    numbers=left,                         % dónde poner los números de línea
    numbersep=5pt,                        % qué tan lejos están los números de línea del código
    showspaces=false,                     % mostrar espacios con subrayados particulares; reemplaza 'showstringspaces'
    showstringspaces=false,               % subrayar los espacios dentro de las cadenas solo
    showtabs=false,                       % mostrar tabulaciones en el código con subrayados particulares
    tabsize=2,                            % tamaños de tabulación a 2 espacios
    language=TeX,                         % lenguaje del código
    morecomment=[l]\#,                    % reconocer # como inicio de comentario en Python
    frame=single,                         % agregar un marco simple alrededor del código
    rulecolor=\color{black}               % color del marco
}

\begin{document}
%----------------------------------------------------------------------------------------
%   PORTADA
%Modificar desde aqui en adelante
%----------------------------------------------------------------------------------------
\begin{titlepage}%Inicio de la carátula, solo modificar los datos necesarios
\newcommand{\HRule}{\rule{\linewidth}{0.5mm}} 
\center 
%----------------------------------------------------------------------------------------
%	ENCABEZADO
%----------------------------------------------------------------------------------------
\includegraphics[width=10cm]{LOGO_UNIVERSIDAD.jpg}\\ % Si esta plantilla se copio correctamente, va a llevar la imagen del logo de la facultad.OBS: Es necesario incluir el paquete: graphicx
\vspace{3cm}
%----------------------------------------------------------------------------------------
%	SECCION DEL TITULO
%----------------------------------------------------------------------------------------
\HRule \\[0.4cm]
{ \huge \bfseries Caso 2}\\[0.4cm]
{ \huge \bfseries Infraestructura en recursos hídricos}\\[0.4cm] % Titulo del documento
{ \huge \bfseries Proyecto de Infraestructura Hidráulica}\\[0.4cm] % Titulo del documento
\HRule \\[1.5cm]
 \vspace{5cm}
%----------------------------------------------------------------------------------------
%	SECCION DEL AUTOR
%----------------------------------------------------------------------------------------
\begin{flushright}
    { \textbf{Profesor:}\\
    Oscar Loyola\\
    \vspace{0.2cm}
    \textbf{Alumnos:}\\
    Bernardo Caprile Canala-Echevarría\\
    Pedro Valenzuela Béjares\\
    Francisco Zegers
    \vspace{0.2cm}

}
\end{flushright}
\vspace{1cm}
%----------------------------------------------------------------------------------------
%	SECCION DE LA FECHA
%----------------------------------------------------------------------------------------
{\large \textbf{\today}}\\[2cm] % El comando \today coloca la fecha del dia, y esto se actualiza con cada compilacion, en caso de querer tener una fecha estatica, reemplazar el \today por la fecha deseada
\end{titlepage}
%----------------------------------------------------------------------------------------
%  INDICE
%----------------------------------------------------------------------------------------
\newpage
\section*{Resumen Ejecutivo}

\newpage
\tableofcontents % Genera el índice automáticamente
\newpage
%----------------------------------------------------------------------------------------

\section*{Introducción}


\newpage

\section{Marco Teórico}

El diseño del sistema de impulsión para el transporte de agua recuperada desde 
el Tranque Ovejería hacia la Planta de Procesos requiere la aplicación integrada 
de principios de mecánica de fluidos, diseño de tuberías, selección de equipos de 
impulsión y normas de ingeniería como ASME B31.4. A continuación, se presenta el 
marco teórico utilizado para el modelamiento hidráulico y estructural del sistema.

% -------------------------------------------------------------------------
\subsection{Caudal y Velocidad de Flujo}

El caudal total requerido para el sistema es:

\[
Q = 0.8\ \text{m}^3/\text{s}
\]

La velocidad media en una tubería de diámetro $D$ se determina mediante:

\[
V = \frac{Q}{A} = \frac{4Q}{\pi D^2}
\]

donde $A$ es el área transversal de la tubería.

% -------------------------------------------------------------------------
\subsection{Número de Reynolds}

El régimen de flujo se caracteriza con el Número de Reynolds:

\[
Re = \frac{V D}{\nu}
\]

donde:
\begin{itemize}
\item $V$: velocidad media del fluido,
\item $\nu$: viscosidad cinemática del agua.
\end{itemize}

% -------------------------------------------------------------------------
\subsection{Factor de Fricción de Swamee--Jain}

Para flujo turbulento, el factor de fricción de Darcy–Weisbach se obtiene con la 
ecuación explícita de Swamee--Jain:

\[
f = \frac{0.25}{\left[ \log_{10}\left( \frac{\varepsilon}{3.7D} 
+ \frac{5.74}{Re^{0.9}} \right) \right]^2}
\]

donde:
\begin{itemize}
\item $\varepsilon$: rugosidad absoluta interna de la tubería,
\item $D$: diámetro interno,
\item $Re$: número de Reynolds.
\end{itemize}

% -------------------------------------------------------------------------
\subsection{Pérdidas por Fricción en Tuberías}

Las pérdidas por fricción de Darcy--Weisbach se determinan mediante:

\[
h_f = f \frac{L}{D} \frac{V^2}{2g}
\]

donde:
\begin{itemize}
\item $L$: longitud del tramo,
\item $g$: aceleración de gravedad.
\end{itemize}

Las pérdidas menores se modelan como:

\[
h_{m} = K \frac{V^2}{2g}
\]

donde $K$ es el coeficiente equivalente de pérdidas locales.

% -------------------------------------------------------------------------
\subsection{Head Total Requerido (TDH)}

El head dinámico total requerido para vencer desnivel y pérdidas es:

\[
H_{\text{total}} = H_{\text{estático}} + \sum h_f + \sum h_m
\]

donde:

\[
H_{\text{estático}} = z_{\text{final}} - z_{\text{inicial}}
\]

% -------------------------------------------------------------------------
\subsection{Curva Característica de la Bomba}

Cada bomba multietapa modelo Goulds 3600 presenta curvas digitizadas de:

\[
H(Q), \qquad \eta(Q), \qquad P(Q)
\]

Estas curvas se interpolaron linealmente entre puntos obtenidos del catálogo.

El head proporcionado por $N$ bombas en serie es:

\[
H_{\text{serie}} = N_{\text{est}} \, H_b
\]

y el caudal por bomba operando en paralelo:

\[
Q_b = \frac{Q_{\text{total}}}{N_{\text{par}}}
\]

La potencia total instalada:

\[
P_{\text{total}} = N_{\text{par}} \, N_{\text{est}} \, P_b
\]

% -------------------------------------------------------------------------
\subsection{Cálculo de NPSHr y NPSHa}

El NPSH requerido proviene del catálogo:

\[
NPSH_{r} = f(Q)
\]

El NPSH disponible en succión se calcula como:

\[
NPSH_{a} = \frac{p_{\text{atm}}}{\rho g} + z_s - h_f - \frac{p_v}{\rho g}
\]

donde:
\begin{itemize}
\item $p_{\text{atm}}$: presión atmosférica,
\item $p_v$: presión de vapor,
\item $z_s$: cota del nivel libre del estanque.
\end{itemize}

La condición para evitar cavitación es:

\[
NPSH_a > NPSH_r
\]

% -------------------------------------------------------------------------
\subsection{Golpe de Ariete (Estimación Simplificada)}

Para un cierre rápido de válvula, la sobrepresión estimada:

\[
\Delta P = \rho a \Delta V
\]

donde:
\begin{itemize}
\item $a$: velocidad de propagación de onda (acero),
\item $\Delta V$: cambio instantáneo en la velocidad.
\end{itemize}

El head equivalente del golpe de ariete:

\[
\Delta H = \frac{\Delta P}{\rho g}
\]

% -------------------------------------------------------------------------
\subsection{Espesor Mínimo según ASME B31.4 (API 5L X65)}

El esfuerzo circunferencial (hoop stress) para tuberías sometidas a presión interna:

\[
\sigma_h = \frac{P D}{2 t}
\]

La norma ASME B31.4 establece que:

\[
\sigma_h \le F_1 S_y
\]

por lo tanto, el espesor requerido es:

\[
t_{\text{req}} = \frac{P D}{2 F_1 S_y}
\]

donde:
\begin{itemize}
\item $P$: presión interna,
\item $D$: diámetro exterior,
\item $S_y$: límite de fluencia del acero X65,
\item $F_1$: factor de diseño (0.72 para oleoductos).
\end{itemize}

La utilización de espesor:

\[
U_t = \frac{t_{\text{req}}}{t_{\text{adoptado}}}
\]

Y la utilización de esfuerzo:

\[
U_{\sigma} = \frac{\sigma_h}{F_1 S_y}
\]

% -------------------------------------------------------------------------
\subsection{Dimensionamiento de Estanques de Amortiguación}

El volumen requerido para un tiempo de autonomía $T$:

\[
V = Q \, T
\]

Asumiendo estanques cilíndricos verticales:

\[
V = \pi \frac{D_t^2}{4} H_t
\]

y despejando diámetro:

\[
D_t = \sqrt{\frac{4V}{\pi H_t}}
\]

donde $H_t$ es la altura útil adoptada del estanque.

% -------------------------------------------------------------------------
\subsection{Línea de Energía (EGL) y Línea Piezo–métrica (HGL)}

En cada punto:

\[
E = z + \frac{p}{\rho g} + \frac{V^2}{2g}
\]

La línea de energía se construye acumulando head estático y pérdidas.

Las estaciones de bombeo agregan head discreto:

\[
E_{\text{post-bomba}} = E_{\text{pre-bomba}} + H_b
\]

% -------------------------------------------------------------------------
\subsection{Interpolación de Curvas del Fabricante}

Para cada parámetro $y = H, \eta, P$, se utilizó interpolación lineal:

\[
y(Q) = y_i + (y_{i+1}-y_i)\frac{Q - Q_i}{Q_{i+1}-Q_i}
\]

con extrapolación lineal controlada fuera de rango.

% -------------------------------------------------------------------------
\subsection{Selección Óptima de Bombas}

Se minimiza:

\[
P_{\text{total}}(N_{\text{par}}, N_{\text{est}})
\]

sujeto a:

\[
N_{\text{est}} H_b \ge H_{\text{total}}
\]

y

\[
Q_{\text{min}} \le Q_b \le Q_{\text{max}}
\]

obtenidos de la curva de la bomba.



\end{document}